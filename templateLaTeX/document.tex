\documentclass[a4paper, 12pt]{article}

\usepackage[brazilian]{babel}
\usepackage[utf8]{inputenc}
\usepackage{amsmath}
\usepackage{indentfirst}
\usepackage{graphicx}
\usepackage{multicol,lipsum}
\usepackage{float}
\usepackage{hyperref}
\usepackage[right = 2cm, left=2cm, top=2.5cm, bottom=3.5cm]{geometry}
\usepackage[table,xcdraw]{xcolor}
\usepackage{fancyhdr}
\pagestyle{fancy}
\fancyhf{}
\lhead{Thiago Duque Saber de Lima}
\chead{TITULO}
\rhead{\today}
\rfoot{\thepage}
\setlength{\headheight}{52pt}

\usepackage{xcolor}
\usepackage{listings}
\usepackage{caption}
\DeclareCaptionFont{white}{\color{white}}
\DeclareCaptionFormat{listing}{%
	\parbox{\textwidth}{\colorbox{gray}{\parbox{\textwidth}{#1#2#3}}\vskip-4pt}}
\captionsetup[lstlisting]{format=listing,labelfont=white,textfont=white}
\lstset{frame=lrb,xleftmargin=\fboxsep,xrightmargin=-\fboxsep}
\usepackage{color} %red, green, blue, yellow, cyan, magenta, black, white
\definecolor{mygreen}{RGB}{28,172,0} % color values Red, Green, Blue
\definecolor{mylilas}{RGB}{170,55,241}

\begin{document}
%\maketitle
\lstset{language=Matlab,%
	%basicstyle=\color{red},
	breaklines=true,%
	morekeywords={matlab2tikz},
	keywordstyle=\color{blue},%
	morekeywords=[2]{1}, keywordstyle=[2]{\color{black}},
	identifierstyle=\color{black},%
	stringstyle=\color{mylilas},
	commentstyle=\color{mygreen},%
	showstringspaces=false,%without this there will be a symbol in the places where there is a space
	numbers=left,%
	numberstyle={\tiny \color{black}},% size of the numbers
	numbersep=9pt, % this defines how far the numbers are from the text
	emph=[1]{for,end,break},emphstyle=[1]\color{red}, %some words to emphasise
	%emph=[2]{word1,word2}, emphstyle=[2]{style},    
}
\begin{titlepage}
	\begin{center}
	
	\begin{figure}[!ht]
	\centering
	\includegraphics[width=0.5\linewidth]{ufjf.png}
	\end{figure}

		\Huge{Universidade Federal de Juiz de Fora}\\
		\large{Faculdade de Engenharia}\\ 
		\large{Graduação em Engenharia Elétrica }\\ 
		\vspace{15pt}
        \vspace{95pt}
        \textbf{\LARGE{TITULO }}\\
		%\title{{\large{Título}}}
		\vspace{3,5cm}
	\end{center}
	
	\begin{flushleft}
		\begin{tabbing}

			Aluno: Thiago Duque Saber de Lima\\
				
			%Professor : 
			
			%Professor co-orientador: \\
	\end{tabbing}
 \end{flushleft}
	\vspace{1cm}
	
	\begin{center}
		\vspace{\fill}
			 Juiz de Fora\\
		 \today
			\end{center}
\end{titlepage}
%%%%%%%%%%%%%%%%%%%%%%%%%%%%%%%%%%%%%%%%%%%%%%%%%%%%%%%%%%%

% % % % % % % % %FOLHA DE ROSTO % % % % % % % % % %

\begin{titlepage}
	\begin{center}
	
	\begin{figure}[!ht]
	\centering
	\includegraphics[width=0.5\linewidth]{ufjf.png}
	\end{figure}

		\Huge{Universidade Federal de Juiz de Fora}\\
		\large{Faculdade de Engenharia}\\ 
		\large{Graduação em Engenharia Elétrica}\\ 
\vspace{15pt}
        
        \vspace{85pt}
        
		\textbf{\LARGE{TITULO}}
		\title{\large{Título}}
	%	\large{Modelo\\
     %   		Validação do modelo clássico}
			
	\end{center}
\vspace{1,5cm}
	
	\begin{flushright}

   \begin{list}{}{
      \setlength{\leftmargin}{4.5cm}
      \setlength{\rightmargin}{0cm}
      \setlength{\labelwidth}{0pt}
      \setlength{\labelsep}{\leftmargin}}

      \item Relatório referente ao "REFERENCIA" da aula de "DISCIPLINA" do Curso de Engenharia Elétrica da Universidade Federal de Juiz de Fora.

      \begin{list}{}{
      \setlength{\leftmargin}{0cm}
      \setlength{\rightmargin}{0cm}
      \setlength{\labelwidth}{0pt}
      \setlength{\labelsep}{\leftmargin}}

			\item Alunos: Thiago Duque Saber de Lima \\ \vspace{15pt}
            \item Professor: NOME DO PROFESSOR \
      		%\item Professor co-orientador: \

      \end{list}
   \end{list}
\end{flushright}
\vspace{1cm}
\begin{center}
		\vspace{\fill}
		 Juiz de Fora\\
		 \today
			\end{center}
\end{titlepage}
\newpage
% % % % % % % % % % % % % % % % % % % % % % % % % %
\newpage
\tableofcontents
\thispagestyle{empty}

\newpage
\pagenumbering{arabic}
% % % % % % % % % % % % % % % % % % % % % % % % % % %
\listoffigures
\newpage
\listoftables
\newpage

\section{Introdução}
\newpage
\section{Exemplo Lslisting}

\begin{lstlisting}[label=Multiprocessamento,caption=Multiprocessamento]
#include <Arduino.h>
#define ledVermelho 22
#define ledVerde 23
int a=10;

void loop2(void*z){
Serial.printf("\nloop2() em core: %d \n", xPortGetCoreID());
while (true){
Serial.println(a);
digitalWrite(ledVerde, !digitalRead(ledVerde));
delay(1000);
}
}

void setup(){
Serial.begin(115200);
pinMode(ledVermelho, OUTPUT);
pinMode(ledVerde, OUTPUT);
Serial.printf("\nsetup() em core: %d", xPortGetCoreID());
xTaskCreatePinnedToCore(loop2, "loop2", 8192, NULL, 1
, NULL, 0);
delay(1);
}

void loop(){
digitalWrite(ledVermelho, !digitalRead(ledVermelho));
delay(1000);
a++;
for i in range
if
}
\end{lstlisting}

\newpage



%\addcontentsline{toc}{section}{Bibliografia}
%\section*{Bibliografia}
%\footnotesize{
%
%\noindent AGUIRRE, L. A. Introdução à Identificação de Sistemas, Técnicas Lineares e Não lineares Aplicadas a Sistemas Reais. Belo Horizonte, Brasil, EDUFMG. 2004.\\

%}
%\newpage
%\addcontentsline{toc}{section}{Anexo}
%\section*{Anexo}
\end{document}


